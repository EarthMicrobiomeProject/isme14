\def\sequence{TACAGAGGATGCAAGCGTTATCCGGAATGATTGGGCGTAAAGCGTCTGTA
GGTGGCTTTTTAAGTCCGCCGTCAAATCCCAGGGCTCAACCCTGGACAGG}
\def\taxonomyGG{k\_\_Bacteria; p\_\_Cyanobacteria; c\_\_Chloroplast; o\_\_Streptophyta; f\_\_; g\_\_; s\_\_}
\def\taxonomyRDP{d\_\_Bacteria; k\_\_; p\_\_Cyanobacteria/Chloroplast; c\_\_Chloroplast; o\_\_; f\_\_Chloroplast; g\_\_Streptophyta (129/129)}
\def\speciesA{Halospirulina sp. (13/129)}
\def\speciesB{Cloning vector (7/129)}
\def\speciesC{Cucumis sativus (5/129)}
\def\wikipedia{Streptophyta or Streptophytina, informally the streptophytes (from the Greek strepto, for twisted, i.e., the morphology of the sperm of some members), is an unranked clade of plants. The composition of the clade varies considerably between authors. One common definition includes the land plants, the embryophytes (bryophytes and vascular plants) and the green algal group Charophyta, which includes the Mesostigmatales, Chlorokybales, Klebsormidiales, Zygnematales, Coleochaetales and Charales.}
\def\prevalencePercent{25.16}
\def\prevalenceRank{3}
\def\abundancePercent{5.488}
\def\abundanceRank{1}
\def\numOTUs{161129}
\def\trimLength{100}
\def\numSamples{1856}
\def\rarefactionDepth{5000}
